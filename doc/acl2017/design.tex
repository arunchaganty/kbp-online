\section{On-demand open-world evaluation}
\label{sec:methodology}

We have just seen that the bias in closed-world evaluation poses an obstacle to innovating new systems.
The fundamental problem was a sampling bias towards relation instances from existing systems.
We could of course completely eliminate the bias by exhaustively annotating the entire document corpus.
Unfortunately, this would be a laborious and prohibitively expensive task:
  using the interfaces we developed (described in detail in \refsec{evaluation}), it costs about \$30 to annotate a single document by non-expert crowdworkers, leading to an estimated cost of at least \$300,000 for a reasonably large corpus of 10,000 documents.
The annotation effort would cost significantly more with expert LDC-style annotation.

In constrast, with on-demand open-world evaluation, we take a lazy approach by annotating underrepresented data \textit{only when required}.
In this section, we cover the technical details that allow us to accurately estimate evaluation metrics without bias in a cost-effective manner. 

% Move to a KBP centric section.
%When a new system or feature has been developed, its predictions are submitted to an evaluation portal.
%If the submitted predictions can not be fairly and accurately evaluated with the data obtained so far,
%  annotations are immediately obtained for a sample of the predicted relations through crowd-sourcing and added into the evaluation set.
%In this way, we push the hard work of finding candidate relations to automated systems and instead focus annotation effort on the much easier task of verifying relation predictions.
%At the end of this process, systems are guaranteed to obtain fair instance-level, relation-level and entity-level scores.

\subsection{Problem statement}

Let us define the task formally.
Let $\sX$ be a universe of possible outputs (e.g.\ relation instances),
  $\sY \subseteq \sX$ be an unknown subset of this universe corresponding to the correct elements in $\sX$ and
  $X_1, \ldots X_m \subseteq \sX$ be known subsets that correspond to the predicted output from $m$ systems.
Furthermore, let $Y_1, \ldots, Y_m$ be the intersection of $X_1, \ldots, X_m$ with $\sY$.
Our goal is estimate the precision, $\pi_i$, and recall, $\rho_i$, of the set of predictions $X_i$ within a certain confidence interval $\epsilon$.
To do so, in on-demand open-world evaluation, we are allowed to ask if $x \in \sY$ (by having crowdworkers validate a system's prediction) or for samples from $\sY$ (by exhaustively annotating a document) at a certain (monetary) cost.

Formally, let $f(x) \eqdef \I[x \in \sY]$ and $g_i(x) = \I[x \in X_i]$, then:
\begin{align*}
  \pi_i  &\eqdef \E_{x \sim X_i}[f(x)] &
  \rho_i &\eqdef \E_{x \sim \sY}[g_i(x)],
\end{align*}
where $x$ is sampled from $X_i$ and $\sY$ according to distributions $p_i(x)$ and $p'(x)$ respectively.
We assume that $p_i(x)$ is known, e.g.\ the uniform distribution over $X_i$, and that samples from $p'(x)$ can be obtained, even if it is unknown.

Clearly, $\pi_i$ and $\rho_i$ can both be estimated by sampling from $X_i$ and $\sY$ respectively.
However, simple statistics tell us that we would require at least 10,000 samples each to estimate $\pi$ and $\rho$ to $\pm 1\%$ point, which can be quite costly on a per-system basis.
We'd like be able to reuse the samples we've collected to evaluate, say $\pi_i$ when evaluating $\pi_j$.
In the rest of this section, we'll see how to do this by answering the following three key questions:
\begin{enumerate}
  \item Suppose we have evaluated $f(x)$ on samples $\Xh_1, \ldots, \Xh_m$ from $X_1, \ldots, X_m$ respectively. How should we best use all of these samples when estimating $\pi_i$?
  \item Can we use the samples $\Xh_1, \ldots, \Xh_m$ when estimating $\rho_i$ in conjunction with samples $\Yh_0$ from $\sY$?
  \item Finally, in practice, we only see the sets $X_1, \ldots, X_m$ sequentially, as they are submitted to the evaluation platform. In this case, how many samples should we draw from $X_m$, given existing samples $\Yh_0$ and $\Xh_1, \ldots, \Xh_{m-1}$?
\end{enumerate}

\subsection{Amortizing costs when estimating precision}

Intuitively, if a set $X_j$ has a significant overlap with $X_i$, we expect that we should be able to its samples when estimating $\pi_i$.
However, it might be case that $X_j$ overlaps with $X_i$ only when $p_i(x)$ is relatively small, in which case the sample $\Xh_j$ is not representative of $X_i$ and a naive combination could lead to the wrong estimate of $\pi_i$.
We address this problem by using ideas from importance sampling \citep{owen}.

In particular, the estimator that we propose is:
\begin{align*}
  \pih_i &= \sum_{j=1}^m w_{ij} \frac{1}{|\Xh_j|} \sum_{x \in \Xh_j} \frac{p_i(x) f(x)}{q_i(x)},
\end{align*}
where $q_i(x) = \sum_{j=1}^m w_{ij} p_j(x)$ and $w_{ij} \ge 0$ are mixture parameters such that $\sum_{j=1}^m w_{ij} = 1$ and $q_i(x) > 0$ wherever $p_i(x) > 0$.
This last condition is easy to guarantee by setting $w_{ii} > 0$.
In \appendixref{sampling} we prove that $\pih_i$ is an unbiased estimator of $\pi_i$ and also work out an expression for its variance. 
This expression of course depends on $f(x)$, but the general intuition is that 
$\pih_i$ will have high variance if $q_i(x) \ll p_i(x)$.
Consequently, we propose choosing $w_{ij} \propto \sum_{x} p_i(x) p_j(x)$, the shared probability mass between $X_i$ and $X_j$.

% TODO: Figure that compares our scheme with equivalent of pooling.
\fake{On simulated experiments on the TAC-KBP dataset, we find a 4-fold decrease in variance using the proposed $\pih$ 
when compared to estimating $\pi_i$ solely using $\Xh_i$}.

\subsection{Amortizing costs when estimating recall}
When estimating recall, we must square the reality that drawing samples from $\sY$ is very expensive because it requires exhaustive annotation: in our experiments we found that it cost roughly 10 times as much to sample from $\sY$ as it does to evaluate $f(x)$.
In contrast, it is very easy to estimate the recall of a system $i$ relative to all the pool of all other submitted systems.
We exploit the fact that the recall of a system $i$, $\rho_i$ can also be expressed as the recall of the system \textit{relative} to the pool, $\nu_i$ and the recall of the pool itself $\theta$: $\rho_i = \theta \nu_i$.

With this in mind, we use a smaller sample $\Yh_0$ from $\sY$ to estimate $\theta$ and then use the rest of $\Xh$ to estimate $\nu_i$.
In \appendixref{sampling} we show that the final estimator, $\rhoh_i \eqdef \thetah \nuh_i$, is unbiased and has a variance of,
\begin{align*}
  \sigma^2_\rho &= \theta \sigma^2_\nu + \nu_i \sigma^2_\theta + \sigma^2_\nu \sigma^2_\theta.
\end{align*}
With sufficient samples, $\sigma^2_\nu$ can be made quite small, so that the leading term in the variance is $\nu_i \sigma^2_\theta$: because typically $\nu_i < 0.5$, we expect the variance of our proposed estimator to be less than that of the estimator constructed by using $\Yh_0$ alone.

% TODO: figure
\fake{On simulated experiments on the TAC-KBP dataset, we find a 2-fold decrease in variance using the proposed estimator $\rhoh$ 
when compared to estimating $\rho_i$ solely using $\Yh_0$}.

\subsection{Adaptively drawing samples for new sets}
A desired property for our framework is to annotate new data only when necessary, i.e.\ a new submission $X_m$ contains sufficiently diverse output.


\subsection{Application to evaluating KBP}
% TODO: include specific details to evaluating on KBP.

%There are two key problems to address in on-demand open-world evaluation.
%%There are two further problems we must address. %three inter-related challenges that we must tackle.
%First, we must ensure that our estimates of recall accurately reflect open world recall despite us not knowing the complete set of true relations.
%% being evaluating on an evaluation set constructed from system predictions.
%% We achieve this by calibrating our scores on a small set of exhaustively annotated documents.
%Secondly, we must ensure that the relations we sample allow us to compute unbiased precision scores.
%%entity-level and relation-level precision scores.
%%  We achieve this through structured sampling.
%%$Finally, we would to implement our evaluation framework cost-effectively.
%%  We achieve this by combining annotations across systems and sampling schemes through importance reweighing.
%We discuss our solutions to these problems next.

%On the other hand, it is much easier to verify a relation instance predicted by a system.
% TODO: outline the structure of the framework better.
% TODO: this should mention pooled annotations, etc.


\subsection{Ensuring open-world recall with exhaustive annotations}
% A line about pooled recall
\pl{I think we should use more formal notation from before to be more precise}

The pooled collection of evaluated relations in the evaluation set is unlikely to be an accurate estimate for all the true relations in the corpus and thus the pooled recall would be a biased estimate.
At the same time, it is impossible to know how many true relations there are in the corpus without annotating it in its entirety.
We address this problem by obtaining exhaustive annotations on a few documents to correct for the bias in using pooled recall.
% vv is possibly too much detail.
%For each sampled document, a crowdworker annotates any entities he/she can find.
%In a separate task, crowdworkers are then asked to identify relations between every pair of mentions in a single sentence.
%%\footnote{%
%%While there certainly are relations across sentences, this requires crowdworkers to evaluate many more mention pairs for relations.}
%During entity annotation, crowdworkers also link entities to Wikipedia pages if possible, allowing us to perform entity-level evaluation.

%We first evaluate whether exhaustive annotations can be reliably performed by crowdworkers.
%To do so, we compare crowdsourced annotations against those of expert annotators using data from the TAC-KBP 2015 EDL task~\citep{}.\footnote{%
%Further details regarding the annotation interface and experiment can be found in \refsec{turk}.
%}
%We find that crowdworkers identify 90\% of the entity spans identified by expert annotators and have significant token-level inter-annotator agreement (\fake{$\kappa = 0.8$}), validating the hypothesis that exhaustive annotation crowdsourced.

Then, when evaluating recall, we first estimate recall of the entire evaluation set on the exhaustively annotated documents, and multiply that by the recall of the system on the evaluation set. Through this two-step process we exploit the fact that it is easier to reduce variance by increasing the size of the evaluation set than it is to increase the size of the exhaustive annotation. \refsec{power} provides more details about this variance reduction.

\pl{Sounds too much like an ad-hoc side note;
this should be part of the problem definition (desired evaluation metric that we're trying to estimate),
not part of our solution; in other words, should be moved to a previous section
}
A final issue to discuss is how documents should be sampled to capture diverse entities that span documents. % to provide better entity-level scores.
When considering uniformly sampled documents, we that find that only extremely frequent entities like the United States or Barack Obama appear across documents.
Unfortunately, we do not know which entities are present in which documents to construct a fairer sampling scheme.
As a heuristic, the 20\% of our exhaustive document collection is sampled uniformly and annotated.
We then uniformly sample the entities annotated to create a collection of ``query entities''.
Finally, we construct the remaining 80\% of our document collection by searching for documents that contain the query entities according to an exact string match.
% TODO: would be nice to quantify 

\subsection{Structured sampling for entity-level and relation-level precision}
Estimation of relation-level and entity-level scores respectively assumes that relation types and entities are sampled uniformly.
However, if we were to uniformly sample relations from a typical system's predictions, 
  we would get a very skewed distribution over relation types and entities and thus bias our estimates.
We address this problem through two structured sampling schemes.

First, for relation-level scores, we stratify the predicted instances by relation and collect the same number of samples from each relation.
Secondly, for entity-level scores,  we first sample subject entities uniformly from the predicted KB, then sample fills for that entity, followed by instances for the fill.
For each such entity, we draw several samples.
% TODO: show a diagram.

Similarly, the distribution over relation instances for each system and scoring scheme (i.e.\ entity-level and relation-level) will be quite different. We correct for bias through importance reweighing~\citep{}.

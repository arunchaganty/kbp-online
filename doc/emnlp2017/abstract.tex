Knowledge base population (KBP) systems take in a large document corpus and extract entities and their relations.
%Thus far, KBP evaluation has relied on datasets constructed based on predictions of existing systems.
Thus far, KBP evaluation has relied on judgements on the pooled predictions of existing systems.
We show that this evaluation is problematic:
when a new system predicts a previously unseen relation, it is penalized even if it is correct,  % PL: want to make this stronger, it cannot be accurately evaluated.
leading to significant bias against new systems, which counterproductively discourages innovation in the field.
%We show that pooling evaluation results in a score that is significantly biased against newly developed systems, discouraging innovation and slowing progress in the field.
Our first contribution is a new importance-sampling based evaluation which corrects for this bias by annotating a new system's predictions on-demand via crowdsourcing.
We show this eliminates bias and reduces variance using data from the 2015 TAC KBP task. % as compared to standard evaluation.
% PL: simulation makes it sound artificial...
Our second contribution is a KBP evaluation service and its annotations, which we make available to the community.
We pilot the service by testing diverse state-of-the-art systems on the TAC KBP 2016 corpus and obtain
%demonstrate that the evaluation permits
%an 
accurate and reliable scores in a cost effective manner. 
%understanding of system performance 
%\pl{can we make this less vague but not longer?}.
%We hope that the service will allow the community to get accurate and timely feedback on exciting new directions in KBP.\@

% Large-scale information extraction systems for knowledge base population (KBP) predict true relations from an unenumerable set of candidate relations within a large document corpus.
% Due to the prohibitive cost of annotating the entire corpus, KBP evaluation has relied on datasets constructed by judging predictions pooled together from existing systems.
% However, when a newly developed system predicts a relation not previously seen, it can not be accurately evaluated.
% We show that this results in an evaluation score that is significantly biased against novel system predictions.
% To address this bias,
%   we introduce on-demand open-world evaluation which continuously expands the evaluation dataset by annotating new system's predictions when necessary through crowdsourcing, while still ensuring that score estimates are unbiased.
% We demonstrate a significant cost savings on simulated experiments from the 2015 TAC-KBP evaluation, and conduct a mock evaluation on the TAC-KBP 2016 document corpus.

%% OLD  vvv
%%\pl{don't like facts since that implies truth - changed to relation, since that's also used later in the paper;
%%be consistent}
%%\pl{added 'pooling bias' to abstract, since that's important!}
%Large-scale information extraction systems for knowledge base population (KBP)
%predict true relations from an unenumerable set of candidate relations from a large document corpus.
%%These predictions have traditionally been evaluated under a closed-world assumption,
%%wherein only a small annotated set of facts are considered to be true and all others false,
%Due to the prohibitive cost of annotating the entire corpus,
%KBP evaluation has been traditionally closed-world,
%wherein only a subset of candidates are annotated, % to be positive or negative,
%and candidates outside this subset are automatically marked negative.
%In the annual TAC-KBP challenge, the subset is constructed from the pooled predictions of previous systems.
%We show that this closed-world evaluation significantly penalizes new system improvements.
%%that make correct predictions outside the reach of previous systems.
%% Using lazy than online evaluation.
%%We break from this paradigm by evaluating newly predicted facts on-demand through crowdsourcing, lazily simulating an open-world.
%To address this bias,
%we introduce a new on-demand evaluation in which a system's predictions are immediately evaluated through crowdsourcing,
%thus lazily simulating an open-world.
%% TODO: Should we mention exhaustive annotation?
%%Moreover, we debias estimates of recall by exhaustively annotating a few randomly sampled documents.
%With additional careful sampling and reweighting,
%%to reduce the cost of evaluation and variance of precision and recall decreases with each additional system evaluated.
%% PL: don't need to talk about variance, since can convert variance reduction into cost reduction
%we are able to produce scores at a fraction of the cost on a mock evaluation of the 2016 TAC-KBP challenge.
%
%%\plg{this is a sentence more for the conclusion}
%%We hope that the combination of a cost-effective automated evaluation with a quick turnaround time along with diverse exhaustive annotations will drive progress in knowledge base population and information extraction as a whole.
%
%%Unfortunately, we find that fixing these problems by constructing a larger ``gold-standard'' dataset is infeasible, costing at least \todo{US\$1,000,000} by our estimates.
%%Instead, we propose and implement a new online evaluation methodology that removes the problem of pooling bias and is able to provide statistically valid results cost effectively through judicious sampling of system submissions and crowdsourcing.
%%A mock evaluation conducted on the 2015 KBP submissions confirms that we are able to provide statistically significant results at about \$100/system.
%%% ARUN: This is getting really wordy.
%%The methodology will be made available as an online evaluation platform that is easy to submit to.
%%%We have implemented this methodology through an online evaluation platform
%%%\todo{sell the benefits of submitting to this challenge?}.
%%We hope this new platform will enable quick progress in the field.
%%\pl{too long}
%
%%We propose opening the closed world for knowledge base populatoin by introducing an online evaluation platform that immediately assesses newly predicted facts through crowdsourcing.
%
%%
%%Large-scale information extraction systems, like knowledge base population, try to predict true facts from a indefinitely large set of candidates in a document corpus.
%%These predictions are typically evaluated under a closed-world assumption, wherein only a small annotated set of facts are considered to be true and all others false, largely due to the prohibitive cost of exhaustively annotating a sufficiently large corpus.
%%%
%%% TODO(DONE?): want to hint that this is a serious problem -- not about "oh, you just have to try harder to get an improvement".
%%% NOTE: I hope that this gets to the point -- new ideas are biased against!
%%We empirically show that the closed world assumption can significantly penalize novel improvements in systems because new predictions are falsely marked negative.
%%% NOTE: The fact that this is TAC-KBP isn't all that important.
%%%using data from the TAC-KBP challenge, an annual competition for knowledge base population.
%%%
%%% TODO: This line is too complex.
%%We break from this paradigm by combining pre-annotated facts with an on-demand sampled evaluation of newly predicted facts through crowdsourcing to provide unbiased estimates of performance in an open-world setting.
%%% TODO: Should we mention exhaustive annotation? (Soln ^^)
%%%NOTE: Eh, both of these are second order points.
%%%Moreover, we debias estimates of recall by exhaustively annotating a few randomly sampled documents.
%%%Through careful sampling, the cost of evaluation and variance of precision and recall decreases with each additional system evaluated.
%%% WHAT ABOUT CHALLENGES?
%%%
%%We are able to obtain scores at a fraction of the cost of and with half the variance of official scores on the 2016 TAC-KBP corpus in a mock evaluation.
%%%
%%% TODO: This line is a bit self-gloaty?
%%We hope that our quick and cost-effective on-demand open world evaluation platform will drive progress in knowledge base population and information extraction as a whole.

\onecolumn
\section{Implementation details}
\label{sec:implementation}

\section{Theoretical proofs for the sampling procedures}
\label{sec:sampling}

Let's recall the notation from \refsec{method}.

Let $\sX$ be a universe of possible outputs (e.g.\ relation instances),
  $\sY \subseteq \sX$ be an unknown subset of this universe corresponding to the correct elements in $\sX$ and
  $X_1, \ldots X_m \subseteq \sX$ be known subsets that correspond to the predicted output from $m$ systems,
  and $Y_1, \ldots, Y_m$ be the intersection of $X_1, \ldots, X_m$ with $\sY$.
Furthermore, 
  let $\Xh_i$ be a set of $n_i$ independent samples drawn from $X_i$ with the distribution $p_i$,
  $\Yh_i$ be the intersection of these sets with $\sY$, and
  $\Yh_0$ be a sample drawn from $\sY$ according to an unknown distribution $p'(x)$.

We would like to evaluate precision, $\pi_i$, and recall, $\rho_i$:
\begin{align*}
  \pi_i  &\eqdef \E_{x \sim X_i}[f(x)] &
  \rho_i &\eqdef \E_{x \sim \sY}[g_i(x)],
\end{align*}

In this section, we'll provide proofs that show that the estimators proposed are indeed unbiased, and we will characterize their variance.

\subsection{Estimating precision}

In \refsec{method}, we proposed the following estimator for $\pi_i$:
\begin{align*}
  \pih_i &\eqdef \sum_{j=1}^m \frac{w_{ij}}{n_j} \sum_{x \in \Xh_j} \frac{p_i(x) f(x)}{q_i(x)},
\end{align*}
where $q_i(x) = \sum_{j=1}^m w_{ij} p_j(x)$ and $w_{ij} \ge 0$ are mixture parameters such that $\sum_{j=1}^m w_{ij} = 1$ and $q_i(x) > 0$ wherever $p_i(x) > 0$.

\begin{theorem}[Statistical properties of $\pih_i$]
\label{thm:pih}
  $\pih_i$ is an unbiased estimator of $\pi_i$ and has a variance of:
  \begin{align*}
    \Var{\pih_i} &= \sum_{j=1}^m \frac{w_j^2}{n_j} \E_{p_j}\left[\frac{{p_i(x)}^2{f(x)}^2 - \pi_{ij}p_i(x)f(x)q_i(x)}{{q_i(x)}^2} \right],
  \end{align*}
  where $\pi_{ij} \eqdef \E_{p_j}\left[\frac{p_i(x)f(x)}{q_i(x)} \right]$.
\end{theorem}
\begin{proof}
  Let $\Xh = (\Xh_1, \ldots, \Xh_m)$ which is drawn from the product distribution of $p_1 \times p_m$.
  By independence and the linearity of expectation, 
  \begin{align*}
    \E_{\Xh}\left[\sum_{j=1}^m f(\Xh_j) \right] &= \sum_{j=1}^m \E_{\Xh_j}[f(\Xh_j)].
  \end{align*}

  First, let's show that $\pih_i$ is unbiased:
  \begin{align*}
    \E_{\Xh}[\pih_i] 
    &= \E_{\Xh}\left[\sum_{j=1}^m \frac{w_j}{n_j} \sum_{x \in \Xh_j} \frac{p_i(x) f(x)}{q_i(x)} \right] \\
    &= \sum_{j=1}^m \frac{w_j}{n_j} \E_{\Xh_j}\left[\sum_{x \in \Xh_j} \frac{p_i(x) f(x)}{q_i(x)} \right] \\
    &= \sum_{j=1}^m \frac{w_j}{n_j} n_j \E_{p_j}\left[\frac{p_i(x) f(x)}{q_i(x)}\right] \\
    &= \sum_{j=1}^m w_j \sum_{x \in \sX} p_j(x) \frac{p_i(x) f(x) }{q_i(x)}\\
    &= \sum_{x \in \sX} \sum_{j=1}^m w_j p_j(x) \frac{p_i(x) f(x) }{q_i(x)}\\
    &= \sum_{x \in \sX} q_i(x) \frac{p_i(x) f(x)}{q_i(x)} \\
    &= \sum_{x \in \sX} p_i(x) f(x) \\
    &= \pi_i.
  \end{align*}

  Now let's compute the variance.
  \begin{align*}
    \Var{\pih_i} &= \sum_{j=1}^m \frac{w_j^2}{n_j} \E_{p_j}\left[\frac{{p_i(x)}^2{f(x)}^2}{{q_i(x)}^2} \right]
    - \sum_{j=1}^m \frac{w_j^2}{n_j} {\E_{p_j}\left[\frac{p_i(x)f(x)}{q_i(x)} \right]}^2 \\
    \Var{\pih_i} &= \sum_{j=1}^m \frac{w_j^2}{n_j} \E_{p_j}\left[\frac{{p_i(x)}^2{f(x)}^2}{{q_i(x)}^2} - \frac{\pi_{ij} p_i(x) f(x)}{q_i(x)} \right] \\
    \Var{\pih_i} &= \sum_{j=1}^m \frac{w_j^2}{n_j} \E_{p_j}\left[\frac{{p_i(x)}^2{f(x)}^2 - \pi_{ij}p_i(x)f(x)q_i(x)}{{q_i(x)}^2} \right],
  \end{align*}
  where $\pi_{ij} \eqdef \E_{p_j}\left[\frac{p_i(x)f(x)}{q_i(x)} \right]$.
\end{proof}

\subsection{Estimating recall}

In \refsec{method}, we used the fact that the recall of system $i$, $\rho_i$, can be expressed as the recall of $i$ within the pool, $\nu_i$ and the recall of the pool itself $\theta$: $\rho_i = \theta \nu_i$:  
\begin{align*}
  \nu_i &= \E_{x \sim \sY \given Y}[g_i(x)] &
  \theta &= \E_{x \sim \sY}[g(x)],
\end{align*}
where $x$ is sampled under the distribution $p'(x \given x \in Y)$ and $p'(x)$ respectively and
  $g(x) \eqdef \I[x \in \Union_{i=1}^m X_i] = \max_{j \in [1,m]} g_j(x)$ is the indicator function for $x$ belonging to the pool.

Ideally, to estimate the pooled recall, $\nu_i$, we need to take expectations with respect to $x \sim Y$, but have samples drawn from individual $X_i$.
To correct for this bias, we'll use a self-normalizing estimator for $\nu_i$:
\begin{align*}
  \nuh_i &\eqdef \frac{\sum_{j=1}^m \frac{w_j}{n_j} \sum_{x \in \Yh_j} \frac{u(x) g_i(x)}{q(x)}}{\sum_{j=1}^m \frac{w_j}{n_j} \sum_{x \in \Yh_j} \frac{u(x)}{q(x)}},
\end{align*}
where $p'(x) \propto u(x)$, $q(x) = \sum_{j=1}^m w_{j} p_j(x)$ and $w_{j} \ge 0$ are mixture parameters such that $\sum_{j=1}^m w_{j} = 1$.

The pool recall $\theta$ can be estimated as follows:
\begin{align*}
  \thetah &\eqdef \sum_{x \in \Yh_0} g(x),
\end{align*}
where $g(x) \eqdef \I\left[ x \in \Union_{i=1}^m X_i \right] = \max_{j \in [1,m]} g_j(x)$.

Finally, we proposed the following estimator for recall $\rho_i$:
\begin{align*}
  \rhoh_i &\eqdef \thetah\nuh_i.
\end{align*}

Let's start by showing that $\nu_i$ is unbiased.
\begin{theorem}[Statistical properties of $\nuh_i$]
\label{thm:nuh}
  $\nuh_i$ is a \textit{consistent} estimator of $\nu_i$.
\end{theorem}
\begin{proof}
  We have that $p'_Y(x) = \frac{w(x)}{Z_Y}$.
  While we do not know the value of $Z_Y$, we can divide both the numerator and denominator of $\nuh_i$ by this quantity:
  \begin{align*}
    \nuh_i 
    &= \frac{\sum_{j=1}^m \frac{w_j}{n_j} \sum_{x \in \Yh_j} \frac{u(x) g_i(x)}{Z_Y q(x)}}{\sum_{j=1}^m \frac{w_j}{n_j} \sum_{x \in \Yh_j} \frac{u(x)}{Z_Y q(x)}} \\
    &= \frac{\sum_{j=1}^m \frac{w_j}{n_j} \sum_{x \in \Yh_j} \frac{p'_Y(x) g_i(x)}{q(x)}}{\sum_{j=1}^m \frac{w_j}{n_j} \sum_{x \in \Yh_j} \frac{p'_Y(x)}{q(x)}}.
  \end{align*}

  As the number of samples $n_i \to \infty$,
  \begin{align*}
    \E_{X}[\nuh_i] 
    &= \E_{X}\left[ \frac{\sum_{j=1}^m \frac{w_j}{n_j} \sum_{x \in \Yh_j} \frac{p'_Y(x) g_i(x)}{q(x)}}{\sum_{j=1}^m \frac{w_j}{n_j} \sum_{x \in \Yh_j} \frac{p'_Y(x)}{q(x)}} \right] \\
    &= \frac{\E_{X}\left[ \sum_{j=1}^m \frac{w_j}{n_j} \sum_{x \in \Yh_j} \frac{p'_Y(x) g_i(x)}{q(x)} \right]}
    {\E_{X}\left[\sum_{j=1}^m \frac{w_j}{n_j} \sum_{x \in \Yh_j} \frac{p'_Y(x)}{q(x)}\right]}.
  \end{align*}

  Following similar arguments as in the proof of \refthm{pih}, the numerator and denominator are unbiased estimators of $\E_{x \sim \sY \given Y}[g_i(x)]$ and $\E_{x \sim \sY \given Y}[1] = 1$ respectively.
  Thus,
  \begin{align*}
    \E_{X}[\nuh_i] 
    &= \E_{x \sim \sY \given Y}[g_i(x)].
    &= \nu_i.
  \end{align*}
  $\nuh_i$ is an unbiased estimator of $\nu_i$.

  % TODO: derive variance which will look ugly as hell.
\end{proof}

Finally, we turn to studying $\rhoh$:
\begin{theorem}[Statistical properties of $\rhoh_i$]
\label{thm:rhoh}
  $\rhoh_i$ is an unbiased estimator of $\rho_i$ with variance
  \begin{align*}
    \Var \rhoh_i &= \theta \Var \nuh_i + \nu_i \Var\thetah + \Var\thetah \Var\nuh_i.
  \end{align*}
\end{theorem}
\begin{proof}
  First, let's show that $\rho_i = \theta \nu_i$:
  \begin{align*}
    \rho_i 
    &\eqdef= \E_{x \sim \sY}[g_i(x)] \\
      &= p'(Y_i) \\
      &= p'(Y \band Y_i) \\
      &= p'(Y) p'(Y_i \given Y) \\
      &= \E_{x\sim \sY}[g(x)] \E_{x \sim \sY\given Y}[g_i(x)] \\
      &= \theta \nu_i.
  \end{align*}


  From \refthm{nuh}, we have that $\nuh_i$ is an unbiased estimator of $\nu_i$.
  It is evident that $\thetah$ is an unbiased estimator of $\theta$.
  $\nuh_i$ and $\thetah$ are estimated using independent samples ($\Yh$ and $\Yh_0$ respectively), and hence
  \begin{align*}
    \E_{Y_0, Y}[\rhoh] 
      &= \E_{Y_0, Y}[\thetah \nuh_i] \\
      &= \E_{Y_0}[\thetah] \E_Y[\nuh_i] \\
      &= \theta \nu_i \\
      &= \rhoh.
  \end{align*}

  By \reflem{variance-product},
  \begin{align*}
    \Var \rhoh_i &= \theta \Var\nuh_i + \nu_i \Var\thetah + \Var\thetah \Var\nuh_i.
  \end{align*}
\end{proof}

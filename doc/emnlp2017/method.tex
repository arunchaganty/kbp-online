\section{On-demand evaluation with importance sampling}
\label{sec:method}

Pooling bias is fundamentally a sampling bias problem where relation instances from new systems are underrepresented in the evaluation dataset.
We could of course sidestep the problem by exhaustively annotating the entire document corpus, by annotating all mentions of entities and checking relations between all pairs of mentions. However, that would be a laborious and prohibitively expensive task:
using the interfaces we've developed (\refsec{evaluation}), it costs about \$15 to annotate a single document by non-expert crowdworkers, resulting in an estimated cost of at least \$1,350,000 for a reasonably large corpus of 90,000 documents \citep{dang2016kbp}.
The annotation effort would cost significantly more with expert annotators.
% TODO: highlight contrast with pooling.
In contrast, \textit{labeling} 
%\pl{be consistent with terminology: labeling?} 
relation instances from system predictions are better than random guess and it
%pooled from system predictions \pl{can we remove 'pooled from system predictions'?} 
can be an order of magnitude cheaper than finding them in documents: using our interfaces, it costs only about \$0.18 to verify each pooled instances compared to \$1.60 per instance extracted through exhaustive annotations.
%\pl{why the diff between 0.18 and 1.60?  Isn't it the same problem of labeling an instance?}

We propose a new paradigm called on-demand evaluation which takes a lazy approach to dataset construction by annotating predictions from systems \textit{only when they are underrepresented}, thus correcting for pooling bias as it arises.
In this section, we'll formalize the problem solved by on-demand evaluation independent of KBP and describe a cost-effective solution that allows us to accurately estimate evaluation scores
%metrics \pl{be consistent: scores} 
without bias using importance sampling.
We'll then instantiate the framework for KBP in \refsec{application}.

\subsection{Problem statement}
Let $\sX$ be the universe of %candidate predictions (e.g.\, relation instances),
(relation) instances,
  $\sY \subseteq \sX$ be the unknown subset of correct instances,
  $X_1, \ldots X_m \subseteq \sX$ be the predictions for $m$ systems,
  and let $Y_1, \ldots, Y_m$ be the intersection of $X_1, \ldots, X_m$ with $\sY$.
Let $X = \Union_{i=1}^m X_i$ and $Y = \Union_{i=1}^m Y_i$.
Let $f(x) \eqdef \I[x \in \sY]$ and $g_i(x) = \I[x \in X_i]$, then the precision, $\pi_i$, and recall, $\rho_i$, of the set of predictions $X_i$ is
\begin{align*}
  %\pi_i  &\eqdef \E_{x \sim X_i}[f(x)] &
  %\rho_i &\eqdef \E_{x \sim \sY}[g_i(x)],
  \pi_i  &\eqdef \E_{x \sim p_i}[f(x)] &
  \rho_i &\eqdef \E_{x \sim p_0}[g_i(x)],
\end{align*}
where $p_i$ is a distribution over $X_i$ and $p_0$ is a distribution over $\sY$.
We assume that $p_i$ is known, e.g.\, the uniform distribution over $X_i$
and that we know $p_0$ up to normalization constant and can sample from it.

In on-demand evaluation, we can query $f(x)$ (e.g.\, labeling an instance) or draw a sample from $p_0$;
typically, querying $f(x)$ is significantly cheaper than sampling from $p_0$.
We obtain prediction sets $X_1, \ldots, X_m$ sequentially as the systems are submitted for evaluation.
Our goal is to estimate $\pi_i$ and $\rho_i$ for each system $i = 1, \dots, m$.

\subsection{Simple estimators}
We can estimate each $\pi_i$ and $\rho_i$ independently with simple Monte Carlo integration. % from $p_i$ and $p_0$ respectively.
Let $\Xh_1, \ldots, \Xh_m$ be $n_1, \ldots, n_j$ i.i.d.~samples from $X_1, \ldots, X_m$ respectively, and let $\Yh_0$ be a set of $n_0$ samples drawn from $\sY$.
Then, the simple estimators for precision and recall are:
\begin{align}
  \pih_i^{\text{(simple)}} &= \frac1{n_i} \sum_{x \in \Xh_i} f(x) & \rhoh_i^{\text{(simple)}} &= \frac1{n_0} \sum_{x \in \Yh_0} g_i(x).
\end{align}

\subsection{Joint estimators}
The simple estimators are unbiased but have wastefully large variance
because evaluating a new system does not leverage labels acquired for previous
systems.  %\paragraph{Estimation Algorithm}

On-demand evaluation with the joint estimator works as follows:
First $\Yh_0$ is randomly sampled from $\sY$ once when the evaluation framework is launched.
For every new set of predictions $X_m$ submitted for evaluation, the minimum number of samples $n_i$ required to accurately evaluate $X_m$ is calculated based on the current evaluation data, $\Yh_0$ and $\Xh_1, \ldots, \Xh_{m-1}$.
Then, the set $\Xh_m$ is added to the evaluation data by evaluating $f(x)$ on $n_m$ samples drawn from $X_m$.
Finally, estimates $\pi_1, \ldots, \pi_m$ and $\rho_1, \ldots, \rho_m$ are updated using the joint estimators.
%requires us to spend money to collect data for every new system submitted.

Thus, we need to answer the following three questions:
\begin{enumerate}
  \item How can we use all the samples $\Xh_1, \ldots \Xh_m$ when estimating the precision $\pi_i$ of system $i$?
  \item How can we use all the samples $\Xh_1, \ldots, \Xh_m$ with $\Yh_0$ when estimating recall $\rho_i$?
  \item Finally, to form $\Xh_m$, how many samples should we draw from $X_m$ given existing samples and $\Xh_1, \ldots, \Xh_{m-1}$ and $\Yh_0$?
\end{enumerate}

\paragraph{Estimating precision jointly.}
Intuitively, if two systems have very similar predictions $X_i$ and $X_j$, we should be able to use samples from one to estimate precision on the other.
However, it might also be the case that $X_i$ and $X_j$ only overlap on a small region, in which case the samples from $X_j$ do not accurately represent instances in $X_i$ and could lead to a biased estimate.
We address this problem by using importance sampling \citep{owen2013monte}, a standard statistical technique for estimating properties of one distribution using samples from another distribution.

In importance sampling, if we sample $x \sim q_i$, then $\frac{p_i(x)}{q_i(x)} f(x)$ is an unbiased estimate of $\pi_i$.
We would like the proposal distribution $q_i$ to both leverage samples from all $m$ systems and be tailored towards system $i$.
To this end, we first define a distribution over systems $j$, represented by probabilities $w_{ij}$.
Then, define $q_i$ as sampling a $j$ and drawing $x \sim p_j$;
formally $q_i(x) = \sum_{j=1}^m w_{ij} p_j(x)$.
There are two refinements:
First, we numerically integrate over all $m$ systems rather than sampling one.
Second, the $n_j$ samples from each $p_j$ are reused for each $i$;
the weights $w_{ij}$ ``mix'' these samples to reduce variance.
Formally, define the \emph{joint precision estimator}:\footnote{We provide rigorous proofs for the
unbiasedness of the proposed estimators in \refapp{sampling} of the
supplementary material.}
\begin{align}
  \pih_i^{\text{(joint)}} &\eqdef \sum_{j=1}^m \frac{w_{ij}}{n_{j}} \sum_{x \in \Xh_j} \frac{p_i(x) f(x)}{q_i(x)},
\end{align}
where each $\Xh_j$ consists of $n_j$ i.i.d.~samples drawn from $p_j$.

%Unfortunately, because $X$ and therefore $q$ changes every time a new set of predictions is submitted, samples drawn earlier are no longer valid.
%At the same time, using a single proposal distribution when estimating different $\pi_i$ could be suboptimal.
% - For example, the proposal distribution could be the weighted mixture over all $X_i$: 
% $q(x) = \frac{1}{m} \sum_{i=1}^m w_i p_i(x)$, for some $w_i > 0, \sum_{i}^m w_i = 1$.
% - However, when a new $X_{m+1}$ is being evaluated, the estimator cannot make use of the samples previously obtained because they would have come from a different proposal distribution.
% - Furthermore, we would prefer reweighting samples when estimated for each $\pi_i$ differently.

% We handle these two desiderata using the following joint precision estimator:
%\ac{Other readers have noted that it might be good to give intuition about this estimator. I'm not sure how to do so given space constraints.}
%We overcome this obstacle by proposing the following joint precision estimator that simulates the behavior of a proposal distribution without actually drawing samples from it:
%\pl{I think a more helpful description would talk about the high-level intuition - and with a figure!  something about suppose you're trying to evaluate $i$,
%how much of other systems should you pull; well, let $w_{ij}$ measure that, and now define $q_i$, and gradually build up rather than defining $\hat \pi_i$ directly}
%\begin{align}
  %\pih_i^{(j)} &= \sum_{j=1}^m \frac{w_{ij}}{n_{j}} \sum_{x \in \Xh_j} \frac{p_i(x) f(x)}{q_i(x)},
%\end{align}
%where $q_i(x) = \sum_{j=1}^m w_{ij} p_j(x)$ and $w_{ij} \ge 0$ are mixture parameters such that $\sum_{j=1}^m w_{ij} = 1$ and $q_i(x) > 0$ wherever $p_i(x) > 0$.
%This last condition is easy to guarantee by setting $w_{ii} > 0$.
%Intuitively, $q_i(x)$ behaves like the marginal distribution over $\sX$ if we drew samples the mixture of $p_j(x)$, with mixture weights $w_{ij}$.\footnote{%

To set the mixing weights $w_{ij}$, we can first formally verify that
if $X_i$ and $X_j$ are disjoint, then $w_{ij} = 0$ minimizes the variance of $\pi_i$,
and if $X_i = X_j$, then $w_{ij} \propto n_{j}$.
%$\pih_i^{(j)}$ will have high variance if $q_i(x) \ll p_i(x)$
%In particular, we can show that the ideal choice \pl{in what sense?} of $w_{ij}$ for
This motivates the following scheme which interpolates between the two extremes:
$w_{ij} \propto n_{j} \sum_{x \in \sX} p_j(x) p_i(x)$.
%=======
%The variance depends on $f(x)$, but the general intuition is that $\pih_i^{(j)}$ will have high variance if 
%$q_i$ does not have tails at least as heavy as $p_i$.
% \pl{$q_i(x) \ll p_i(x)$ what does this mean? there exists an $x$ such this is true? you're comparing functions}.
%In particular, we can show that the ideal choice \pl{in what sense?} of $w_{ij}$ for if $X_i$ and $X_j$ are disjoint is $0$, and if $X_i$ and $X_j$ are identical is $w_{ij} \propto n_{j}$.
%This motivates the choice $w_{ij} \propto n_{j} \sum_{x \in \sX} p_j(x) p_i(x)$.

\paragraph{Estimating recall jointly.}
During the process of evaluating $f(x)$ on $\Xh_i$, we also are able to identify $\Yh_i \eqdef \Xh_i \intersection \sY$.
The collection of $\Yh = \Union_{i=1}^m \Yh_i$ forms a random subset of $\sY$ that is much cheaper to obtain than by sampling from $\sY$.
% PL: should be obvious
%but this subset can be extremely biased if used to evaluate recall:
%if there were one system, it would obtain 100\% recall regardless of how bad it was.
We call this the \emph{incomplete estimate pooled recall},
$\nu_i \eqdef \E_{x \sim X}[f(x) g_i(x)]$.
We must also consider how much of $\sY$ the pool $X$ actually covers, i.e.\ the pool recall: $\theta \eqdef \E_{x \sim \sY}[g(x)]$ where $g(x) \eqdef \I[x \in X]$.
Thus, the recall of a system is simply the product: $\rho_i = \theta \nu_i$.\footnote{%
We present a rigorous proof in \refapp{sampling} of the supplementary material.}

The joint estimator for recall is then $\rhoh_i^{\text{(joint)}} \eqdef \thetah\nuh_i$, where $\thetah$ is $\sum_{x \in \Yh_0} g(x)$ and $\nuh_i$ is a self-normalized importance-weighted estimator:
\begin{align*}
  \nuh_i^{\text{(joint)}} &\eqdef \frac{\sum_{j=1}^m \frac{w_{ij}}{n_j} \sum_{x \in \Xh_j} \frac{u(x) g_i(x)}{q_i(x)}}{\sum_{j=1}^m \frac{w_{ij}}{n_j} \sum_{x \in \Xh_j} \frac{u(x)}{q_i(x)}},
\end{align*}
where $q_i$ and $w_{ij}$ are the same as defined for $\pih_i^{\text{(joint)}}$.

\paragraph{Adaptively choosing the number of samples.}
Finally, a desired property for on-demand evaluation is to label new instances only when the current evaluation data is insufficient,
e.g.\ when a new set of predictions $X_m$ contains many instances not covered by other systems.
We can measure how well the current evaluation set covers the predictions $X_m$ by using an upper bound on the variance of $\pih_m^{\text{(joint)}}$.\footnote{Details can be found in \refapp{sampling} of the supplementary material.}
In particular, the variance of $\pih_m^{\text{(joint)}}$ is a monotonically decreasing function in the number of samples of $X_m$ drawn $n_m$ and we can easily solve for the minimum number of samples required to estimate $\pih_m^{\text{(joint)}}$ within a confidence interval $\epsilon$ by using the bisection method \citep{burden1985bisection}.
